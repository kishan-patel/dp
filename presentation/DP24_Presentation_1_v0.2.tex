\documentclass{beamer}
\setbeamertemplate{navigation symbols}{}
\usepackage{beamerthemeshadow}

\begin{document}

\title[Dashboard for MAB Algorithms]{Dashboard for Multi Armed Bandit (MAB) Algorithms}
\author[Surbhi Gupta, Kishan Patel]{Surbhi Gupta, Kishan Patel}
\date{November 13, 2013}

\begin{frame}
\titlepage
\begin{center}
Supervisor: Aditya Mahajan, Design Project 1
\end{center}
\end{frame}

\begin{frame}
\tableofcontents
\end{frame}

\section{Overview}

\subsection{Objective and Purpose}
\begin{frame}{Objective and Purpose}
\textbf{Objective}
\newline To build a \textbf{dashboard} in order to represent the results of executing a generic class of \textbf{Multi Armed Bandit} (MAB) algorithms used for \textbf{Website Optimization} (WO) 
\newline
\\\textbf{Purpose}
\\Ease of identification of best performing (most efficient) MAB algorithm for WO as well as
\begin{itemize}
  \item In-depth visual understanding
  \item Engaging interactive design
\end{itemize}
\end{frame}

\subsection{Terminology}
\begin{frame}{Terminology}
Some terms to familiarize with 
\begin{itemize}
  \item \textbf{Agent}: Decision maker
  \item \textbf{Arm}: Action 
  \item \textbf{Gain}: Measure of success or reward
\end{itemize}
\end{frame}

\subsection{MAB Problem and Algorithm}
\begin{frame}{MAB Problem}
\textbf{Problem}
\newline An \textbf{agent} chooses 1 \textbf{arm}, and receives a \textbf{gain} from it.
\newline How can the agent \textbf{maximize} his gain?
\newline
\newline \textbf{Algorithm}
\newline Look for the most optimal arm by
\begin{itemize}
  \item Exploiting the highest performing arms
  \item Exploring other arms to see if they perform even better
\end{itemize}
\end{frame}

\subsection{Website Optimization}
\begin{frame}{Website Optimization}
WO as a bandit problem
\newline
\newline What do each of these represent?
\begin{itemize}
  \item Agent: User
  \item Arm: Website version with unique
	\begin{itemize}
		\item Color scheme
		\item Layouts
		\item Size of buttons
	\end{itemize}
  \item Gain: \textbf{Effectiveness} of a particular website version
	\begin{itemize}
		\item Effectiveness can be defined as a metric of success 
		\item Definition varies across different domains
		\item Eg. 1 Number of purchases of a particular item on Amazon.com
		\item Eg. 2 Number of donors on a fundraising website
	\end{itemize}
\end{itemize}
\end{frame}

\section{Progress till date}

\subsection{Implementation Overview}
\begin{frame}{Implementation Overview}
\begin{enumerate}
	\item Introduction to MAB
	\item Model WO as a MAB problem
	\item Identify the purpose of a creating a dashboard 
	\begin{itemize}
		\item Research to choose a suitable charting library
		\item Create graphs using the chosen library
		\item Discuss feedback with supervisor
	\end{itemize}
	\item Next steps
	\begin{itemize}
		\item Prioritize requirements and visit backlog
		\item Create a tentative timeline for next semester
	\end{itemize}
\end{enumerate}
\end{frame}

\subsection{Charting Library Research}
\begin{frame}{Charting Library Research}
	\begin{itemize}
		\item Options explored: Radian, Cubism.js, NVD3.js, Rickshaw
		\item Narrowed choices to: Radian, Rickshaw
	\end{itemize}
\end{frame}

\begin{frame}{Radian}
\begin{tabular}{| p{2.5cm} | p{8cm} |}
    \hline
     \textbf{Parameter} & \textbf{Radian} \\ \hline
  \textbf{Reliability} \newline & In development phase\newline Released in 2013 (very new) \\ \hline
  \textbf{Resource \newline Availability} &  Well organized tutorial documentation
		\newline External resources for Angular.js directives 
		\newline Untidy and non-intuitive Github repository \\ \hline
  \textbf{Learning Curve} &  Knowledge of HTML
		\newline Custom HTML elements can represent functional and data plots
		\newline Angular.js knowledge for interactive plots \\ \hline
\textbf{Features and Extensibility}& Limited basic features (covered by Rickshaw) \\ \hline
\end{tabular}
\end{frame}

\begin{frame}{Rickshaw}
\begin{tabular}{| p{2.5cm} | p{8cm} |}
    \hline
     \textbf{Parameter} & \textbf{Rickshaw} \\ \hline
\textbf{Reliability}\newline & Established framework\newline Released in 2011 \\ \hline
\textbf{Resource \newline Availability} &  Limited and concise tutorial documentation
		\newline Comprehensive '/examples' section in Github repository\\ \hline
\textbf{Learning Curve}  &  Knowledge of JavaScript for functional, data and interactive plots \\ \hline
\textbf{Features and Extensibility} & Feature rich
		\newline Vast range of extensions to build on and extend existing functionality\\ \hline
\end{tabular}
\end{frame}

\begin{frame}{Charting Library Research: Result}
\begin{itemize}
	\item Final choice: Rickshaw
	\item Increased reliability- more established framework
	\item Enhanced resource availability- comprehensive Github repository
	\item Neutral learning curve
		\begin{itemize}
			\item Common skill between group members- JavaScript
			\item Limited time to learn a new framework (Angular.js)
		\end{itemize}
	\item Rich feature set
		\begin{itemize}
			\item Wide range of extensions suitable for our project
			\item Eg. Time fixture feature for incorporating time series graphs
		\end{itemize}
\end{itemize}
\end{frame}

\subsection{Viewing File Data}
\begin{frame}{Viewing File Data}
TODO: Explain what this is about and put a screenshot
\end{frame}

\subsection{Viewing When a Particular Arm is Played}
\begin{frame}{Viewing When a Particular Arm is Played}
TODO: Explain what this is about and put a screenshot
\end{frame}

\subsection{Viewing Results by Time}
\begin{frame}{Viewing Results by Time}
TODO: Explain what this is about and put a screenshot
\end{frame}

\subsection{Running a Particular Simulation on Known Data}
\begin{frame}{Running a Particular Simulation on Known Data}
TODO: Explain what this is about and put a screenshot
\end{frame}

\section{Future Plans}

\subsection{Support for Live Data}
\begin{frame}{Support for Live Data}
TODO: Explain what this is about
\end{frame}

\subsection{Enhance Interactivity}
\begin{frame}{Enhance Interactivity}
TODO: Explain what this is about
\end{frame}

\section{Methodology}

\subsection{Organization and Challenges}
\begin{frame}{Organization and Challenges}
TODO: 
Organization: Talk about how the group organizes itself (meetings breakdown, how we communicate)
\newline Challenges: Talk about the challenges faced eg. in picking charting library, learning curve 
\end{frame}

\end{document}