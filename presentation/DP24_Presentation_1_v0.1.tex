\documentclass{beamer}
\setbeamertemplate{navigation symbols}{}
\usepackage{beamerthemeshadow}

\begin{document}

\title[Dashboard for MAB Algorithms]{Dashboard for Multi Armed Bandit (MAB) Algorithms}
\author[Surbhi Gupta, Kishan Patel]{Surbhi Gupta, Kishan Patel}
\date{November 13, 2013}

\begin{frame}
\titlepage
\begin{center}
Supervisor: Aditya Mahajan, Design Project 1
\end{center}
\end{frame}

\begin{frame}
\tableofcontents
\end{frame}

\section{Overview}

\subsection{Objective and Purpose}
\begin{frame}{Objective and Purpose}
\textbf{Objective}
\newline To represent the results of executing a generic class of
MAB algorithms used for Website Optimization (WO)
\newline
\\\textbf{Purpose}
\\Ease of identification of best performing (most efficient) MAB algorithm for WO as well as
\begin{itemize}
  \item In-depth visual understanding
  \item Engaging interactive design
\end{itemize}
\end{frame}

\subsection{Terminology}
\begin{frame}{Terminology}
Some terms to familiarize with 
\begin{itemize}
  \item \textbf{Agent}: Decision maker
  \item \textbf{Arm}: Action 
  \item \textbf{Gain}: Measure of success or reward
\end{itemize}
\end{frame}

\subsection{MAB Problem and Algorithm}
\begin{frame}{MAB Problem}
\textbf{Problem}
\newline An \textbf{agent} chooses 1 \textbf{arm}, and receives a \textbf{gain} from it.
\newline How can the agent \textbf{maximize} his gain?
\newline
\newline \textbf{Algorithm}
\newline Look for the most optimal arm by
\begin{itemize}
  \item Exploring new or existing arms
	\begin{itemize}
		\item Existing arms to see if they perform even better
	\end{itemize}
  \item Exploiting the high performing arms
\end{itemize}
\end{frame}

\subsection{Website Optimization}
\begin{frame}{Website Optimization}
WO as a bandit problem
\newline
\newline What do each of these represent?
\begin{itemize}
  \item Agent: User
  \item Arm: Website version with unique
	\begin{itemize}
		\item Color scheme
		\item Layouts
		\item Size of buttons
	\end{itemize}
  \item Gain: Effectiveness of a particular website version
\end{itemize}
\end{frame}

\section{Progress till date}

\subsection{Charting Library Research}
\begin{frame}{Charting Library Research}
	\begin{itemize}
		\item Options explored: Radian, Cubism.js, NVD3.js, Rickshaw
		\item Narrowed choices to: Radian, Rickshaw
	\end{itemize}
\begin{tabular}{| l | l | l |}
    \hline
     & Radian & Rickshaw \\ \hline
  1 & 2 & 3  \\
  4 & 5 & 6  \\
  7 & 8 & 9  \\
\end{tabular}
\end{frame}

\subsection{Viewing File Data}
\begin{frame}{Viewing File Data}
TODO: Explain what this is about and put a screenshot
\end{frame}

\subsection{Viewing When a Particular Arm is Played}
\begin{frame}{Viewing When a Particular Arm is Played}
TODO: Explain what this is about and put a screenshot
\end{frame}

\subsection{Viewing Results by Time}
\begin{frame}{Viewing Results by Time}
TODO: Explain what this is about and put a screenshot
\end{frame}

\subsection{Running a Particular Simulation on Known Data}
\begin{frame}{Running a Particular Simulation on Known Data}
TODO: Explain what this is about and put a screenshot
\end{frame}

\section{Future Plans}

\subsection{Support for Live Data}
\begin{frame}{Support for Live Data}
TODO: Explain what this is about
\end{frame}

\subsection{Enhance Interactivity}
\begin{frame}{Enhance Interactivity}
TODO: Explain what this is about
\end{frame}

\section{Methodology}

\subsection{Organization and Challenges}
\begin{frame}{Organization and Challenges}
TODO: 
Organization: Talk about how the group organizes itself (meetings breakdown, how we communicate)
\newline Challenges: Talk about the challenges faced eg. in picking charting library, learning curve 
\end{frame}

\end{document}